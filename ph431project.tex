\documentclass{article}
\usepackage{amsmath}
\usepackage[margin=1in]{geometry}
\usepackage{graphicx}
\usepackage{fancyhdr}
\setlength\parindent{12pt}
\pagestyle{fancy}

\begin{document}
\lhead{Properties and Uses of Optical Vortices}
\rhead{Ellefson, Grant, Thieme, and Zeto}
\chead{}
\title{Properties and Uses of Optical Vortices}
\author{Steven Ellefson, David Grant, Mattson Thieme, Rene Zeto\\\\PH 431 Final Project}
\maketitle

\section*{Background}
Optical vortices have been shown to possess unique characteristics which make them potentially useful in a number of different applications. An optical vortices, or a light beam containing a ``screw dislocations," is a light wave with a helical structure such that perfect destructive interference occurs at its center, forming a node$^{1}$. When projected onto a surface, an optical vortex will appear as a ring of light with a dark node at its center. The structure and appearance of an optical vortex are shown in Figure 1 below.

\begin{figure}[hc]
		\centering
			%\includegraphics[height=1.5in]{}
			\caption{Hey Rene, I couldn't find the cool diagram you showed us of both the structure of an optical vortex and the projection onto a surface appearance. Could you put that in here? }
		\end{figure}
		
A comprehensive paper in 1974 by Nye and Berry$^{2}$ summarized the theory for the different experimentally-observed dislocations that can be induced in light wave trains, including screw dislocations. An extraordinary characteristic of optical vortices is the photons in the beams carry a conserved orbital angular momentum around the beam's center that is independent of the photons' intrinsic angular momentum. Information can thus be carried in two independent and distinct channels in an optical vortex. The helical shape of the beam can also exert torques when focused onto objects. Accordingly, these beams can be useful for applications in secure communication, particle traps referred to as ``optical tweezers," and in quantum computing.\\\\\\

From Steve: So I've at least started our paper and put up a skeleton Powerpoint for our presentation Monday. The paper isn't due until Monday night so we don't have to completely have it done before our presentation. Mattson, I know you have done some research into uses of optical vortices; I'm not sure what you're planning for the presentation, but I'm willing to be the first speaker to talk about research and uses. Since you've done quite a bit, though, you can of course speak instead of me if you want. Just let me know. I do not have class before E and M on Monday and I'm getting to school at about 7, so I'll just be working on our presentation up until class starts. My plan for the paper and the first part of the presentation was to talk about the first experimental observation of optical vortices in the 70's, mention a few potential applications, and then select two applications and present at least one to two papers about each of them. Please let me know what you guys need me to do further. I want to pull my weight for this project. I can finalize our presentation Monday morning, as I'll have a couple hours before class. Just let me know what needs to be done and I'll finish it. Thanks, guys!



\section*{References}
\begin{enumerate}
	\item{Sundbeck, Steven; Gruzberg, Ilya; and Grier, David G. ``Structure and Scaling of  Helical Modes of Light." Optics Letters, Vol. 30, Issue 5, pp. 477-479 (2005)}
	\item{Nye, J. F.; and M. V. Berry (1974). ``Dislocations in wave trains" (PDF). Proceedings of the Royal Society of London, Series A 336 (1605): 165. (1974)}
\end{enumerate}

\end{document}
